\input{Math Club W10/Packages}
\input{Math Club W10/Definitions}

\fancypagestyle{firstpageheader}{
    \fancyhf{}
% Header
    \fancyhead[L]{\MathClubW{11}}
    \fancyhead[R]{12/2/2024}
% Footer
    \fancyfoot[C]{\thepage}
    \renewcommand{\headrulewidth}{0.4pt} 
}

\parindent 0pt
\marginparsep 3pt

\clubpenalty=500 % Control break penalty for single lines at top
\widowpenalty=500 % Control break penalty for single lines at bottom
\displaywidowpenalty=500
\interlinepenalty=0 % Encourage breaks between lines

\begin{document}

\sloppy
\thispagestyle{firstpageheader}

\Sect{pain}


\begin{problem}[D][7][Putnam 2014 A1]
Prove that every nonzero coefficient of the Taylor series of 
\[
(1 - x + x^2)e^x
\]
about $x = 0$ is a rational number whose numerator (in lowest terms) is either $1$ or a prime number.
\end{problem}

\begin{solution}
    Using $\displaystyle e^x=\sum_{n=0}^\infty \frac{x^n}{n!}$ gives us
    \begin{align*}
        (1 - x + x^2)e^x \ = \ (1 - x + x^2)\sum_{n=0}^\infty \frac{x^n}{n!} \ = \ \sum_{n=0}^\infty \frac{x^n}{n!} - \sum_{n=0}^\infty \frac{x^{n+1}}{n!} + \sum_{n=0}^\infty\frac{x^{n+2}}{n!}
    \end{align*}
    Then re-indexing, pulling out terms so they all start from $n=2$, and combining the sums, we get
    \begin{align*}
        (1 - x + x^2)e^x &= \sum_{n=0}^\infty \frac{x^n}{n!} - \sum_{n=1}^\infty \frac{x^{n}}{(n-1)!} + \sum_{n=2}^\infty\frac{x^{n}}{(n-2)!}\\
        &= 1 + \sum_{n=2}^\infty \left[\frac{x^n}{n!} - \frac{x^{n}}{(n-1)!} + \frac{x^{n}}{(n-2)!}\right]\\
        &= 1 + \sum_{n=2}^\infty \frac{x^n(n-1)^2}{n(n-1)(n-2)!} = 1 + \sum_{n=2}^\infty \frac{(n-1)}{n(n-2)!}x^n
    \end{align*}
    And so for $n \geq 2$ our coefficients are $c_n=\frac{(n-1)}{n(n-2)!}$. There are two possibilities: $n-1$ is or is not prime.
    
    If $n-1$ is prime, the conditions are obviously met since the numerator is prime.
    
    If $n-1$ is not prime, then it can be written as the product of two (prime or composite) factors $k_1$ and $k_2$ where $1 < k_1,\, k_2 < n-1$. And since these will also be factors of $(n-2)!$ by the definition of factorials, we know $k_1k_2 \mid n(n-2)! \Rightarrow n-1 \mid n(n-2)!$ meaning the numerator is a factor of the denominator. So in lowest terms, the $n-1$ would cancel and we'd be left with a numerator of 1. So the conditions are met. $\Box$
\end{solution}

\begin{problem}[D][9][Putnam 2023 A3]
     Determine the smallest positive real number $r$ such that there exist differentiable functions $f : \mathbb{R} \to \mathbb{R}$ and $g : \mathbb{R} \to \mathbb{R}$ satisfying:
    \begin{enumerate}
        \item $f(0) > 0$,
        \item $g(0) = 0$,
        \item $|f'(x)| \leq |g(x)|$ for all $x$,
        \item $|g'(x)| \leq |f(x)|$ for all $x$, and
        \item $f(r) = 0$.
    \end{enumerate}
\end{problem}

\begin{solution}[$\pi/2$]
    \textbf{Based on the official one}\\
   Well, conditions (3) and (4) give an interesting relation between $f$ and $g$ that looks like the sine and cosine functions. In fact if we let $f(x)=\cos x$ and $g(x) = \sin x$, all conditions are satisfied, leaving $r=\pi /2$. So we are motivated to prove that we always have $r \geq \pi / 2$.\\
   But it looks like we should force a trigonometric function to get $\pi$ from somewhere, an interesting way is by trying to achieve something like $\frac{1}{x^2+1}=\frac{d}{dx}\tan^{-1}x$.\\ \vspace{5pt}
   \begin{align*}
        &\left | \frac{d}{dx} \frac{g(x)}{f(x)} \right |= \left |\frac{g'(x)f(x)-   g(x)f'(x)}{f^2(x)}\right | \leq \frac{|g'(x)f(x)|+|g(x)f'(x)|}{f^2(x)} \leq \frac{f^2(x)+g^2(x)}{f^2(x)} \\
        \Rightarrow &\left | \frac{d}{dx} \frac{g(x)}{f(x)} \right | \leq 1+\frac{g^2(x)}{f^2(x)} \Rightarrow \frac{d}{dx}\tan^{-1} \left( \frac{g(x)}{f(x)} \right ) \leq 1
    \end{align*}
    Then 
    \begin{align}
        \int_0^r \left[\frac{d}{dx}\tan^{-1} \left( \frac{g(x)}{f(x)} \right )\right]dx \leq \int_0^r [1] dx = r
    \end{align}
    We can't directly evaluate $g(r)/f(r)$ since $f(r)=0$, but we can see that $\tan (g/f) \rightarrow \pi/2$ as $x \rightarrow r^-$, then (1) becomes
    \begin{align*}
        \ \frac{\pi}{2} - \tan^{-1} \left( \frac{g(0)}{f(0)} \right ) \leq r \\
        \Rightarrow \frac{\pi}{2} \leq r
    \end{align*}
    Then we are done...well, no. We are not\\
    For $\tan (g/f) \rightarrow \pi/2$ to be true we have to make sure that $g(r) \neq 0$, because otherwise $\lim_{x \rightarrow r^{-1}}g(r)/f(r)$ would have an indeterminate form. To show $g(r) \neq 0$ we use again $f^2(x)+g^2(x)$:\\
    \begin{align*}
        &\frac{d}{dx} \left | f^2(x)+g^2(x) \right| = |2f(x)f'(x)+2g(x)g'(x)| \leq |2f(x)g(x)| + |2g(x)f(x)| \\
        \Rightarrow&\frac{d}{dx} \left | f^2(x)+g^2(x) \right| \leq 4|f(x)g(x)| \leq 2f^2(x)+2g^2(x) \text{\hspace{20pt}by AM-GM}\\
    \end{align*}
    Now we connected the expression with its derivative. If we can show that $f^2(x)+g^2(x)$ is non-decreasing, then $f^2(0)+g^2(0)=f^2(0)>0 \Rightarrow f^2(r)+g^2(r)=g^2(r) >0$. In an ideal world we would be able to do so, successfully finishing the problem. We do not live in an ideal world, but we can prove something cooler (keeping in mind that $|a| \geq -a$):
    \begin{align*}
        &2f^2(x)+2g^2(x)+\frac{d}{dx}\left[ f^2(x)+g^2(x) \right] \geq 0 \\
        \iff &e^{2x} \left( 2f^2(x)+2g^2(x)+\frac{d}{dx}\left[ f^2(x)+g^2(x) \right] \right) \geq 0 \\
        \iff &\frac{d}{dx} \left [ e^{2x}(f^2(x)+g^2(x)) \right ] \geq 0
    \end{align*}
    Then $e^{2x}(f^2(x)+g^2(x))$ is non decreasing. It is greater than zero at $x=0$, so it cannot be zero at $x=r$ and hence $g(r) \neq 0$ $\Box$
\end{solution}

\begin{problem}[D][10][Putnam 2022 A1]
    Determine all ordered pairs of real numbers $(a, b)$ such that the line $y = ax + b$ intersects the curve $y = \ln(1 + x^2)$ in exactly one point.
\end{problem}
\vskip 3mm
\textbf{Official Solution}\\
    $(a,b)=(0,0)$. If $|a| \geq 1$ then any $b \in \mathbb{R}$, and if $a \in (0,1)$ then any $b$ such that $b < \ln^2(1-x_-)-|a|x_-$ or $b> \ln^2(1-x_+)-|a|x_+$. Where
    $$ x_\pm = \frac{1\pm \sqrt{1-a^2}}{a}$$
    Let $f(x)=\ln(x^2+1)-ax$, we will assume $a \geq 0$ since $\ln(x^2+1)$ is an even function and any solution for $a$ is part of a valid pair if an only if $-a$ is part of a valid pair too.\vspace{7pt} \\
    Let's start with the case $(a,b)=(0,0)$ by showing $a=0 \Rightarrow b=0$. Here $f(x)=\ln(x^2+1)=f(-x)$, so if we want a unique solution for $f(x)=b$, it should occur when $x=-x \Rightarrow x=0 \Rightarrow 0=f(0)=b$ \vspace{7pt} \\
    Next, we claim that if $a \geq 1$ then $f$ always has only one zero for any $b$.\\
    For $x$ small enough, $ax+b$ becomes negative, so $\ln(x^2+1)\geq 0 > ax+b$ and
    for $x$ large enough, we have $e^{ax+b}>x^2+1 \iff ax+b>\ln(x^2+1)$. Since both are continuous functions, they must intersect at least once. Now suppose that they intersect more than once, assume the existence of $x_1 \neq x_2$ such that $f(x_1)=f(x_2)=b$.
    \begin{align*}
    x^2+1 \geq 2x \Rightarrow 1 \geq \frac{2x}{x^2+1}  \\
    0 \geq 1-a \geq \frac{2x}{x^2+1}-a \Rightarrow 0 \geq f'(x) \\ 
    0=f(x_2)-f(x_1)=\int_{x_1}^{x_2} f'(x)dx < 0 
    \end{align*}
    Since there are infinitely many cases for which the equality case does not hold. Hence there's a unique $x$ for which $f(x)=0$. \vspace{7pt} \\
    Now, if $a \in (0,1)$, we have that $f'(x_\pm)=0$. So $f$ is increasing on $(x_-,x_+)$ and decreasing anywhere else. So if $b \in (f(x_-),f(x_+))$ then it should have three solutions, two if $b=f(x_\pm)$ and one otherwise $\Box$


\begin{problem}[D][7][Putnam 2022 B1]
    Suppose that $P(x) = a_1x + a_2x^2 + \cdots + a_nx^n$ is a polynomial with integer coefficients, with $a_1$ odd. Suppose that $e^{P(x)} = b_0 + b_1x + b_2x^2 + \cdots$ for all $x$. Prove that $b_k$ is nonzero for all $k \geq 0$.
\end{problem}

\begin{solution}
    We see from the very beginning that this has something to do with Taylor Series. A first attempt may be that, knowing $e^x=1+x/1!+x^2/2!+...$. We get that
    \begin{align*}
        e^{P(x)} = \prod_{i=1}^n \left(\sum_{k=0}^{\infty} \frac{x^{ik}}{k!} \right)^{a_i}
    \end{align*}
    And this is the best way for not solving the problem.\\
    Instead of that, we can try to get all the derivatives of $f(x)=e^{p(x)}$ at $x=0$. Because for all $k \geq 0$ we have that $b_k$ is $f^{(k)}(0)/k!$, based on the Taylor Series of $f(x)$ centered at $x=0$.\\
    
    \begin{align*}
        &\frac{d}{dx} e^{P(x)} = e^{P(x)} \left(P(x)+ P'(x) \right)\\
        \Rightarrow &\frac{d}{dx} e^{P(x)} = e^{P(x)} \left(P(x)+ P'(x) + P''(x)\right)\\
    \end{align*}
    Here we can see a pattern, in fact we can use those examples as base cases to prove by induction something interesting:\vspace{12pt} \\
    \textbf{Claim: }  $\forall k \geq 0: f^{(k)}(x) = e^{P(x)}(P(x)+P'(x)+\ldots+P^{(k)}(x))$\\
    Proof (by induction):
    \begin{align*}
        f^{(k+1)}(x) = \frac{d}{dx}f^{(k)}(x) =& \frac{d}{dx} \left[ e^{P(x)}\left (P(x)+P'(x)+\ldots+P^{(k)}(x) \right) \right] \\
        =& e^{P(x)}\left (P(x)+P'(x)+\ldots+P^{(k+1)}(x) \right) \\
        \Box
    \end{align*}
    While $k \leq n$ we have that $P^{(k)}(0)=k!a_k$, and $P^{(k)}(0)=0$ for $k > n$, this means that 
    \begin{align*}
        f^{(k)}(0) =
        \begin{cases} 
            1 & k=0\\
            a_1+2a_2+\ldots+k!a_k &  1\leq k \leq n \\ 
            a_1+2a_2+\ldots+n!a_n &  k > n.
        \end{cases}
    \end{align*}
    Note that for $k\geq 2$, $k!$ is even, and since $a_1$ is odd, $f^{(k)}(0)$ is odd, so it's not zero. $f^{(k)}(0) \neq 0$ directly means $b_k \neq 0$, hence we are done $\Box$
\end{solution}

\begin{problem}[D][8][Putnam 2016 B1]
        Let $x_0, x_1, x_2, \dots$ be the sequence such that $x_0 = 1$ and for $n \geq 0$,
        \[
        x_{n+1} = \ln\left(e^{x_n} - x_n\right)
        \]
        (as usual, the function $\ln$ is the natural logarithm). Show that the infinite series
        \[
        S = x_0 + x_1 + x_2 + \cdots
        \]
        converges and find its sum.
\end{problem}

\begin{solution}
   Observe that raising $e$ to both sides of the recursive equation gives
   \begin{align}\setcounter{equation}{0}
       e^{x_{n+1}}=e^{x_n}-x_n
   \end{align}
   Then, replacing $n$ with $n+1$ we get
   \begin{align*}
        e^{x_{n+2}} &= e^{x_{n+1}}-x_{n+1}\\
        \Rightarrow e^{x_{n+2}} &= e^{x_n}-x_n-x_{n+1}
   \end{align*}
   Doing this again gives $e^{x_{n+3}} = e^{x_n}-x_n-x_{n+1}-x_{n+2}$. Notice that the terms at the end are beginning to resemble the desired sum. If we continue this pattern and plug in $n=0$, we get
   \begin{align*}
        e^{x_{n+k}} = e^{x_n}-(x_n+x_{n+1}+\dots+x_{n+k-1}) \quad \Rightarrow \quad e^{x_{k}} = e^{x_0}-(x_0+x_{1}+\dots+x_{k-1})
   \end{align*}
   Finally, taking the limit of both sides as $k \rightarrow \infty$, we have
      \begin{align*}
        \lim_{k\rightarrow\infty}(e^{x_{k}}) &= e^{x_0}-S
   \end{align*}
   where $S$ is the desired sum. So we just need $\lim_{k\rightarrow\infty}(e^{x_{k}})$ which we can find by taking the limit of both sides of (1) as $n\rightarrow\infty$ and letting $x=\lim_{n\rightarrow\infty}x_n$ so we have 
   \begin{align*}
       e^x=e^x-x \quad \Rightarrow \quad x=0
   \end{align*}
   Thus $\lim_{k\rightarrow\infty}(e^{x_{k}}) = e^0 = 1$. And since we $x_0=1$ was given, we are pretty much done.
   \begin{align*}
       S &= e^{x_0}-\lim_{k\rightarrow\infty}(e^{x_{k}})\\
       &= e^1 - 1 = \boxed{e-1}
   \end{align*}
\end{solution}

\begin{problem}[A][9][Putnam 2023 A2]
    Let $n$ be an even positive integer. Let $p$ be a monic, real polynomial of degree $2n$; that is to say, 
    \[
    p(x) = x^{2n} + a_{2n-1}x^{2n-1} + \cdots + a_1x + a_0
    \]
    for some real coefficients $a_0, a_1, \ldots, a_{2n-1}$. Suppose that $p(1/k) = k^2$ for all integers $k$ such that $1 \leq |k| \leq n$. Find all other real numbers $x$ for which $p(1/x) = x^2$.
\end{problem}

\begin{solution}[$\pm 1/n!$]
    By the Fundamental Theorem of Algebra we know that a monic polynomial is uniquely determined by its roots. Here we don't know the roots of $p(x)$ but we know that $p(1/k)=k^2$ for exactly $2n$ values ($|k| \leq n$). 
    \begin{align*}
        &p \left( \frac{1}{x}\right) = \frac{1}{x^{2n}} + \frac{a_{2n-1}}{x^{2n-1}}+\ldots+ \frac{a_1}{x}+a_0 \\
        \iff &x^{2n} \cdot p\left( \frac{1}{x}\right) = a_0x^{2n}+a_1x^{2n-1}+\ldots+a_{2n-1}x+1 
    \end{align*}
    Let $g(x) = x^{2n+2}-x^{2n}p(1/x)$, since $x^{2n}p(1/x)$ is a polynomial, $g(x)$ is also a polynomial, in particular, a monic polynomial. Then $g(\pm k) = 0$, so $g$ is of the form
    \begin{align*}
        g(x) = (x^2-1)(x^2-4)\cdots(x^2-n^2)q(x)
    \end{align*}
    For some monic, quadratic polynomial $q(x)$. Let $q_1,q_2$ be the roots of $q$, then we know by Vietta's Theorem that:
    \begin{align*}
        1+2+\ldots+n-1-2-\ldots-n +q_1+q_2=0 \iff q_1=-q_2 \\
        (-1)^{n+1}\cdot 1^2 \cdot2^2 \cdots n^2 \cdot q_1^2 = -(-1)^{2n+2} \\
    \end{align*}
    Recall that $n$ is even, so $ (-1)^{n+1}=-1$ and
    \begin{align*}
        q_1 = \frac{1}{n!} , q_2=-\frac{1}{n!}
    \end{align*}
    This directly means that if $x=\pm 1/n!$ then $p(1/x)=x^2$, we found all the $2n+2$ roots of $g(x)$ so these are all the solutions $\Box$
\end{solution}

\end{document}